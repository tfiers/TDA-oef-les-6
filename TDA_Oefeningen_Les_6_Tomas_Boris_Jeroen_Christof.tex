\documentclass[10pt]{beamer}
%evt, als problemen met fonts:
%\documentclass[mathserif]{beamer}
\usefonttheme[onlymath]{serif}
\usepackage[latin1]{inputenc}
\usepackage[dutch]{babel}
\usepackage{amsmath}
\usepackage{amsfonts}
\usepackage{amssymb}
\usepackage{graphicx}

\title{Oefeningen Les 6}
\subtitle{Quoti\"{e}ntstructuur en groep-homomorfisme, hfst 3.5}
\author{{\scriptsize Tomas Fiers, Boris Gordts, Jeroen Hermans, Christof Luyten}}
\date{maart 2014}
\institute{{\normalsize Toegepaste Discrete Algebra}}

\begin{document}

\begin{frame}
	\titlepage
\end{frame}

\begin{frame}
	\frametitle{Oefening 2.47}
	\begin{block}{Opgave}
		Beschouw de groepen $\langle \mathbb{Z}_{4}, + \rangle$
		en $\langle \mathbb{Z}_{5}\setminus\lbrace0\rbrace,\:\cdot\: \rangle$.\\
		Zoek isomorfismen tussen deze twee groepen.
	\end{block}
	\begin{block}{Oplossing}
		
	\end{block}
\end{frame}

\begin{frame}
	% Optionele oefening
	\frametitle{Oefening 2.71}
	\begin{block}{Opgave}
	\begin{itemize}
		\item Zoek een niet-triviaal groepshomomorfisme tussen 
		enerzijds de verzameling van de reguliere $2\times2$ binaire matrices 
		over het veld $\lbrace0,1\rbrace$
		met de vermenigvuldiging van matrices en anderzijds 
		de permutaties van drie elementen.
	\end{itemize}
	\end{block}
	\begin{block}{Oplossing}
		
	\end{block}
\end{frame}
\begin{frame}
	% Optionele oefening
	\frametitle{Oefening 2.71}
	\begin{block}{Opgave}
	\begin{itemize}
		\item Bestudeer de orde van beide elementen in beide groepen.
	\end{itemize}
	\end{block}
	\begin{block}{Oplossing}
		
	\end{block}
\end{frame}
\begin{frame}
	% Optionele oefening
	\frametitle{Oefening 2.71}
	\begin{block}{Opgave}
	\begin{itemize}
		\item Leg verbanden tussen deze ordes en het groep homomorfisme
	\end{itemize}
	\end{block}
	\begin{block}{Oplossing}
		
	\end{block}
\end{frame}

\begin{frame}
	\frametitle{Oefening 2.65}
	\framesubtitle{(2.65.1)}
	\begin{block}{Opgave}
		Veronderstel dat $H$ een deelgroep is van de multiplicatieve groep $G$ 
		en $K$ een normaaldeler van $G$.
		\begin{itemize}
			\item Bewijs dat $HK = \lbrace hk \mid h \in H 
			\text{ en } k \in K\rbrace$ een deelgroep is van $G$ 
			en dat $K$ een normaaldeler is van $HK$
		\end{itemize}
	\end{block}
	\begin{block}{Oplossing}
		
	\end{block}
\end{frame}
\begin{frame}
	\frametitle{Oefening 2.65}
	\framesubtitle{(2.65.2)}
	\begin{block}{Opgave}
		\begin{itemize}
			\item Toon aan dat de afbeelding $f : H \rightarrow HK/K$
			waarbij $h \rightarrow KH$ een homomorfisme is 
			van $H$ op $HK/K$.
		\end{itemize}
	\end{block}
	\begin{block}{Oplossing}
		
	\end{block}
\end{frame}
\begin{frame}
	\frametitle{Oefening 2.65}
	\framesubtitle{(2.65.3)}
	\begin{block}{Opgave}
		\begin{itemize}
			\item Toon aan dat $\text{Ker}\,f = H \cap K$. Leg ook uit dat 
			$H/(H \cap K)$ isomorf is met $HK/K$.
		\end{itemize}
	\end{block}
	\begin{block}{Oplossing}
		
	\end{block}
\end{frame}

\begin{frame}
	\frametitle{Oefening 2.69}
	\begin{block}{Opgave}
		Beschouw de groep $G = \langle
		\mathbb{Z}_{13} \setminus \lbrace0\rbrace,\;\cdot\;\rangle$ 
		van de gehele getallen modulo 13 zonder nul 
		en met de vermenigvuldiging modulo 13.
		\begin{itemize}
			\item Bepaal de orde van al de elementen.
		\end{itemize}
	\end{block}
	\begin{block}{Oplossing}
		
	\end{block}
\end{frame}
\begin{frame}
	\frametitle{Oefening 2.69}
	\begin{block}{Opgave}
		\begin{itemize}
			\item Is de groep commutatief?
		\end{itemize}
	\end{block}
	\begin{block}{Oplossing}
		
	\end{block}
\end{frame}
\begin{frame}
	\frametitle{Oefening 2.69}
	\begin{block}{Opgave}
		\begin{itemize}
			\item Zoek een groepshomomorfisme van $G$ naar 
			$\langle K,\;\cdot\;\rangle$,
			met $K$ de verzameling van de complexe getallen 
			met absolute waarde 1.
		\end{itemize}
	\end{block}
	\begin{block}{Oplossing}
		
	\end{block}
\end{frame}
\begin{frame}
	\frametitle{Oefening 2.69}
	\begin{block}{Opgave}
		\begin{itemize}
			\item Pas op dit homomorfisme de ontbinding 
			van het groepshomomorfisme toe.
		\end{itemize}
	\end{block}
	\begin{block}{Oplossing}
		
	\end{block}
\end{frame}

\begin{frame}
	\frametitle{Oefening 2.83.2}
	\begin{block}{Opgave}
		Beschouw de verzameling $U(14)$ van de natuurlijke getallen zonder nul 
		en kleiner dan 14, die onderling ondeelbaar zijn met 14. Geef aan
		deze verzameling de inwendige samenstellingswet van
		de vermenigvuldiging modulo 14.
		\begin{itemize}
			\item Ga na dat $U(14)$ isomorf is 
			met de productgroep $U(7)\times U(2)$
		\end{itemize}
	\end{block}
	\begin{block}{Oplossing}
		
	\end{block}
\end{frame}

\end{document}